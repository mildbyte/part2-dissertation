\documentclass[12pt,a4]{article}
\usepackage{hyperref}
\usepackage{color}
\usepackage{fullpage}
\begin{document}

\vfil


\begin{flushright}
\large{A. I\v{s}kovs}\\
\texttt{ai280}\\
Trinity College
\end{flushright}

\vspace{0.3in}
\begin{center}{\large Computer Science Tripos, Part II Project Proposal}

\vspace{0.3in}
\textbf{\center{\Large Predicting drug-pathway interactions using the Correlated Topic Model}}

\vspace{0.4in}
\centerline{\large \today}
\end{center}

\vfil

\noindent{\bf Project Originator:} N. Pratanwanich
\vspace{0.4in}

\noindent
{\bf Project Supervisor:} N. Pratanwanich\\

\noindent
{\bf Signature:}
\vspace{0.2in}

\noindent
{\bf Director of Studies:} Dr A. C. Norman\\

\noindent
{\bf Signature:}
\vspace{0.2in}
 
\noindent
{\bf Project Overseers:} Dr~A.~V.~S.~Madhavapeddy  \& Dr~S.~H.~Teuffel\\

\noindent
{\bf Signatures:}

\pagebreak

% Main document

\section*{Introduction}

Latent Dirichlet Allocation\cite{Blei} is a bag-of-words topic modelling algorithm that treats documents in a corpus as being generated by picking a set of topics and then drawing words from those topics. Using a sampling method, the posterior distributions of unknown variables (probability distributions of topics and words within topics) are inferred. This allows, for an arbitrary document, to infer a probability distribution of topics which it is about.

The 2014 paper\cite{Pratanwanich2014} by Naruemon Pratanwanich and Pietro Lio describes a model based on this algorithm to predict drug-pathway relationships: differential gene expression profiles of drug treatments are treated as documents, pathways where genes are functionally grouped as topics and genes as words. The model had to be augmented with priors: the pathway-gene relationships for some pathways are already known. In the end, this method provided better predictive results of pathways activated by certain drugs than the state-of-the art methods.

A problem with Latent Dirichlet Allocation and hence this approach is that it assumes that the topics (or pathways) are uncorrelated. This assumption is usually not true: a document about Computer Science is more likely to be about Mathematics than, say, Geology. Another 2014 paper\cite{C4MB00014E} by Naruemon Pratanwanich and Pietro Lio presents a model that assumes pathway crosstalk (correlation). It uses matrix factorisation together with this assumption to predict pathway responsiveness for drugs that affect multiple pathways and finds using the correlation assumption results in a better fit to the gene expression data.

The Correlated Topic Model\cite{2007} was proposed as an improvement on the Latent Dirichlet Allocation: instead of using the Dirichlet distribution to model the relative proportions of topics in a document, it uses the logistic normal distribution. The parameters of the distribution are the mean and the covariance matrix, which allows for modelling correlations between topics. The original paper\cite{2007} uses a deterministic approach called Variational EM (Expectation Maximization) to train this model (due to the non-conjugacy of the logistic normal, the Markov Chain Monte Carlo approach with Gibbs Sampling is intractable). The Correlated Topic Model fit the sample corpus (a set of articles from \textit{Science}) better than LDA\cite{2007}.

Since the matrix factorization model that assumed pathway correlation\cite{C4MB00014E} performed better than the existing methods and the LDA method\cite{Pratanwanich2014} improved on this, it is suspected that adapting the CTM to the problem of prediction of pathway responsiveness to drug treatment would yield another improvement in predictive power.

The aim of this project is to implement and test such a model. In addition to reusing the Bayesian network from the CTM, the model will be augmented with a prior distribution of known gene-pathway relationships. The equations for training the model will be derived, based on the existing CTM equations for Variational EM\cite{2007}. The model will be tested on a synthetic corpus of drug gene expression data and then on publically available datasets (CMAP\cite{CMap} for gene expression data and KEGG\cite{KEGG} for pathway data). The results will be compared with the latent variables inferred by the LDA approach. While it's not possible to guarantee that the model will perform better in every case, the results will be investigated to see in which cases the LDA approach performs better than CTM and vice versa.

\section*{Starting Point}

I know Python and have some experience in working with the SciPy stack. Python has lots of facilities for data processing and while it is an interpreted language, NumPy uses native LAPACK/BLAS (linear algebra) libraries as a backend. This means that the performance of the linear algebra routines is on par with C.

Since this project uses some concepts from the Artificial Intelligence II course (see Key Concepts), I will have to familiarise myself with the course before it begins.

There exists an R package for the Correlated Topic Model, as well as a C implementation of the Model written by the authors of the original paper (\url{https://www.cs.princeton.edu/~blei/ctm-c/}). I will however only be able to use those as a reference at most: adding priors to the model will change its implementation. In addition, the existing library implementations also don't have a way to use the fitted models for prediction of pathways perturbed by an unknown drug, which is one of the extensions to the project.

\section*{Substance and Structure of the Project}

\subsection*{Key Concepts}

The key concepts in this project are drawn from the Artificial Intelligence II course, including topics such as Bayesian Networks and inference, methods for training and evaluating classifiers and probability distribution transformation. This project, however, will go beyond the scope of the course, since the Markov Chain Monte Carlo approach for inference in Bayesian networks is intractable for the Correlated Topic Model and Variational Expectation Maximization has to be used.

\subsection*{Major work items}

\subsubsection*{Required reading and research}
This involves familiarizing myself with the Bayesian Networks using the Artificial Intelligence II lecture notes, understanding the Bayesian Network training methods that are used in Latent Dirichlet Allocation (Gibbs Sampling and Markov Chain Monte Carlo) and the Correlated Topic Model (Variational EM). I will also need to get a better understanding of the biological context around the project, including topics such as genes, pathways, how the drug gene expression data is obtained using microarrays and how it has to be preprocessed.

\subsubsection*{Developing a model}

Understand and modify the existing Variational EM equations used to fit the CTM to a corpus so that the model can incorporate prior information.

\subsubsection*{Implementation and testing}

Implement the software to generate synthetic data for model verification and preprocess real datasets. At this stage, the actual training procedures for the model will be implemented in Python.

During development, the model will be tested using toy datasets of synthetic gene expression and pathway data (see Evaluation and Success Criteria).

\subsubsection*{Extensions}
The original CTM paper does not mention how the CTM can be used to calculate the topic proportions for a new document. One extension would hence involve performing that, in which case it will be possible to compare the performance of the model with the LDA approach.

\textit{Note that without this, we are still able to compare with LDA by the inferred latent variable (topic-document distribution).}

There are multiple other priors that can be incorporated into the model to improve predictive power, for example, the correlation between drugs: the molecular structure of a drug can be treated as a set of functional groups and so drugs with similar sets of functional groups are likely to have similar effects.

In addition, the Variational EM approach only gives a maximum likelihood estimate of the model parameters. The model could be given a full Bayesian treatment by using hierarchical Bayesian modelling. \textbf{todo}

\subsubsection*{Fallback plan}
In case I fail to incorporate priors into the Correlated Topic Model, I will use one of the available libraries for fitting the CTM and instead work on processing the gene data into a drug "document" that can be understood by the library, as well as developing a way to predict pathway distributions for a previously unseen drug. The evaluation procedure in that case will have to be changed, since not having priors in the model will mean that the pathways will be completely arbitrary and will not correspond to real drug data.

\section*{Evaluation and Success Criteria}

The main criterion for success is a working implementation of the model as per its specification. This will be evaluated as follows:
\begin{itemize} 
\item Generate a toy dataset of random pathways, pathway-gene memberships (a probability distribution over genes) and drug-pathway interactions (a probability distribution over pathways)
\item Use that to generate a corpus of gene expression data as per how the Correlated Topic Model assumes it's generated.
\item Together with a random subset of the pathway-gene membership data to serve as a prior, use the model to infer the latent variables and compare them with the ones that generated the dataset.
\end{itemize}

Note that the successful achievement of this objective does not imply the model has any predictive power for real drug data. It only verifies that the model works as expected and there are no errors in the implementation.

The comparison with the LDA method will be performed in case the main objective is achieved. \textbf{discussion, compare with LDA on reference data} Then a method for predicting pathways activated by a certain drug will developed as an extension. The model will be tested on reference data (known drug-pathway relationships) by performing the predictions for known drugs and then using the pathway ranking metric described in\cite{Pratanwanich2014}, in which case the performance of the two models can be compared.

\section*{Resource Declaration}

The data for pathways will be taken from the KEGG\cite{KEGG} database and the drug gene expression data will be taken from CMAP\cite{CMap}, which are publically available.

I will be using my own laptop for main development, a quad-core machine with Windows and ArchLinux on it. The source code will be kept in a Git repository which will be regularly pushed to GitHub, as well as to my MCS filespace. As the KEGG and CMAP data is publically available, and the code used to read and prepare it for processing will be backed up, I do not expect to require to make backups of the actual datasets.

\section*{Timetable: Workplan and Milestones to be achieved.}

Planned starting date is 27/10/2014.

\begin{enumerate}

\item {\bf 27/10/2014 -- 09/11/2014} 

Perform required reading and research (Graphical Models and inference in them, LDA, generative models, CTM). Set up the outline for the dissertation.

\item {\bf 10/11/2014 -- 23/11/2014} 

Design the model and adapt the equations used for inference to incorporate priors. Start writing the Method section of the dissertation.

\item {\bf 24/11/2014 -- 11/01/2015} 

Implement the model in Python. Generate a small toy dataset to verify the implementation. Debug the code.

\item {\bf 12/01/2015 -- 25/01/2015}

Work on optimizing and debugging the code so that it can run in reasonable time on a large toy dataset. Write the progress report. Work on the Method and the Evaluation sections of the dissertation.

\textbf{Milestone:} a working implementation of the model. 

\item {\bf 26/01/2015 -- 08/02/2015} 

Submit the progress report. Rehearse and deliver a presentation to the overseeing group.

Obtain and explore the KEGG and the CMAP datasets. Preprocess the data into a suitable format for the model.

\item {\bf 09/02/2015 -- 22/02/2015}

Derive the equations required for prediction of pathways that are perturbed by a drug (extension 1).

\item {\bf 23/02/2015 -- 08/03/2015}

Use the CMAP/KEGG datasets to train the model and evaluate it against the performance of the LDA model.

\item {\bf 08/04/2015 -- 19/04/2015}

Work on the extensions to the project (if there is time) and continue writing the dissertation.

\item {\bf 20/04/2015 -- 03/05/2015}

Finish writing a draft dissertation. Submit to the Director of Studies and the Supervisor for review, allowing 2 weeks to read the draft.

\item {\bf 04/05/2015 -- 10/05/2015}

Incorporate comments from the reviews. Submit the dissertation.

\end{enumerate}

\bibliographystyle{unsrt}
\bibliography{bibliography}

\end{document}