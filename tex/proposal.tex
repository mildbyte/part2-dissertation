\documentclass[12pt,a4]{article}
\usepackage{url}
\usepackage{fullpage}
\begin{document}

\vfil

\centerline{\Large Project Proposal}
\vspace{0.4in}
\centerline{\Large Computer Science Tripos, Part II}
\vspace{0.4in}
\centerline{\Large Predicting drug-pathway interactions using the Correlated Topic Model }
\vspace{0.4in}
\centerline{\large A. I\v{s}kovs, Trinity College}
\vspace{0.3in}
\centerline{\large Originator: Dr P. Li\`{o}}
\vspace{0.3in}
\centerline{\large 17$^{th}$ October 2014}

\vfil


\noindent
{\bf Project Supervisor:} Dr P. Li\`{o}
\vspace{0.2in}

\noindent
{\bf Director of Studies:} Dr A. C. Norman
\vspace{0.2in}
\noindent
 
\noindent
{\bf Project Overseers:} Dr~A.~V.~S.~Madhavapeddy  \& Dr~S.~H.~Teuffel

\pagebreak

% Main document

\section*{Introduction, The Problem To Be Addressed}

Latent Dirichlet Allocation\cite{Blei} is a bag-of-words topic modelling algorithm that treats documents in a corpus as being generated by picking a set of topics and then drawing words from those topics. Then, the posterior distributions of parameters (probability distributions of topics and words within topics) are found.

The 2014 paper\cite{Pratanwanich2014} by Naruemon Pratanwanich and Pietro Lio used this algorithm to predict drug-pathway relationships by treating drugs as documents, pathways as topics and genes as words. The model had to be augmented with priors: the pathway-gene relationships for some pathways are already known. In the end, this method provided better predictive results than the state-of-the art methods.

A problem with Latent Dirichlet Allocation and hence this approach is that it assumes that the topics (or pathways) are uncorrelated. This assumption is usually not true: a document about Computer Science is more likely to be about Mathematics than, say, Geology. 

The Correlated Topic Model\cite{2007} addresses this issue by modelling the topic selection as drawn from a multivariate normal distribution with a certain covariance matrix (which defines the correlations). Unfortunately, this makes the Gibbs sampling approach to optimize the model used in LDA intractable. Hence, variational inference has to be used. However, the Correlated Topic Model fit the sample corpus (a set of articles from {\em Science}) better than LDA.

It is hence expected that adapting the Correlated Topic Model to drug-pathway relationships will yield an improvement in predictive power, after it is as well augmented with a prior distribution of known gene-pathway relationships.

\section*{Starting Point}

I know Python and have some experience in working with the SciPy stack. Python has lots of facilities for data processing and while it is an interpreted language, NumPy uses native LAPACK/BLAS (linear algebra) libraries as a backend. This means that the performance of the linear algebra routines is on par with C.

Since this project uses some concepts from the Artificial Intelligence II course (see Key Concepts), I will have to familiarise myself with the course before it begins.

There exists an R package for the Correlated Topic Model, which might help me with my implementation. A C implementation of the Model, written by the authors of the original paper, also exists (\url{https://www.cs.princeton.edu/~blei/ctm-c/}). I will however only be able to use those as a reference at most: adding priors to the model will change its implementation.

\section*{Substance and Structure of the Project}

\subsection*{Key Concepts}

The key concepts in this project are drawn from the Artificial Intelligence II course, including topics such as Bayesian Networks and inference, methods for training and evaluating classifiers and probability distribution transformation. This project, however, will go beyond the scope of the course, since the Markov Chain Monte Carlo approach for inference in Bayesian networks is intractable for the Correlated Topic Model and Variational Expectation Maximization is used.

\section*{Evaluation and Success Criteria}
During development, the model will be evaluated using standard machine learning techniques: having a training set and a test set. k-fold cross-validation will also be used.

{\bf Main criterion:} The implemented model should have better predictive power for drug-pathway relationships than the LDA approach ({\em measure?})

{\bf Extension:} There are multiple other priors that can be incorporated into the model to improve predictive power, for example, the correlation between drugs: the molecular structure of a drug can be treated as a set of functional groups and so drugs with similar sets of functional groups are likely to have similar effects.

\section*{Resource Declaration}

The data for pathways will be taken from the KEGG database and the drug gene expression data will be taken from CMAP, which are publically available.

I will be using my own laptop for main development, a quad-core machine with Windows and ArchLinux on it. The source code will be kept in a Git repository which will be regularly pushed to GitHub, as well as to my MCS filespace. As the data is publically available, I do not expect to require to make backups of it.

\section*{Timetable: Workplan and Milestones to be achieved.}

Planned starting date is 27/10/2014.

\begin{enumerate}

\item {\bf Michaelmas weeks 2-4} Learn to use X. Read book Y. Read papers Z.

\item {\bf Michaelmas weeks 5-6} Do preliminary test of Q.

\item {\bf Michaelmas weeks 7-8} Start implementation of main task A.

\item {\bf Michaelmas vacation} Finish A and start main task B.

\item {\bf Lent weeks 0-2} Write progress report. Generate corpus of test examples. Finish task B.  

\item {\bf Lent weeks 3-5} Run main experiments and achieve working project.

\item {\bf Lent weeks 6-8} Second main deliverable here.

\item {\bf Easter vacation:} Extensions and writing dissertation main chapters.

\item {\bf Easter term 0-2:}  Further evaluation and complete dissertation.

\item {\bf Easter term 3:} Proof reading and then an early submission so as to concentrate on examination revision.

\end{enumerate}



\bibliographystyle{plain}
\bibliography{bibliography}

\end{document}