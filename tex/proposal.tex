\documentclass[12pt,a4]{article}
\usepackage{hyperref}
\usepackage{color}
\usepackage{fullpage}
\begin{document}

\vfil


\begin{flushright}
\large{A. I\v{s}kovs}\\
\texttt{ai280}\\
Trinity College
\end{flushright}

\vspace{0.3in}
\begin{center}{\large Computer Science Tripos, Part II Project Proposal}

\vspace{0.3in}
\textbf{\center{\Large Predicting drug-pathway interactions using the Correlated Topic Model}}

\vspace{0.4in}
\centerline{\large \today}
\end{center}

\vfil

\noindent{\bf Project Originator:} N. Pratanwanich
\vspace{0.4in}

\noindent
{\bf Project Supervisor:} N. Pratanwanich\\

\noindent
{\bf Signature:}
\vspace{0.2in}

\noindent
{\bf Director of Studies:} Dr A. C. Norman\\

\noindent
{\bf Signature:}
\vspace{0.2in}
 
\noindent
{\bf Project Overseers:} Dr~A.~V.~S.~Madhavapeddy  \& Dr~S.~H.~Teuffel\\

\noindent
{\bf Signatures:}

\pagebreak

% Main document

\section*{Introduction}

Latent Dirichlet Allocation\cite{Blei} is a bag-of-words topic modelling algorithm that treats documents in a corpus as being generated by picking a set of topics and then drawing words from those topics. Using a sampling method, the posterior distributions of unknown variables (probability distributions of topics and words within topics) are inferred. This allows, for an arbitrary document, to infer a probability distribution of topics which it is about.

The 2014 paper\cite{Pratanwanich2014} by Naruemon Pratanwanich and Pietro Lio describes a model based on this algorithm to predict drug-pathway relationships:  drugs are treated as documents, pathways as topics and genes as words. The model had to be augmented with priors: the pathway-gene relationships for some pathways are already known. In the end, this method provided better predictive results of pathways activated by a certain drug than the state-of-the art methods.

A problem with Latent Dirichlet Allocation and hence this approach is that it assumes that the topics (or pathways) are uncorrelated. This assumption is usually not true: a document about Computer Science is more likely to be about Mathematics than, say, Geology. 

The Correlated Topic Model\cite{2007} addresses this issue by modelling the topic selection as drawn from a multivariate normal distribution with a certain covariance matrix (which defines the correlations). Unfortunately, this makes the stochastic Gibbs sampling approach to train the model used in LDA intractable. Hence, an approach called Variational Inference has to be used, which is deterministic and biased (the resultant posterior distribution has systematic errors). Despite that, the Correlated Topic Model fit the sample corpus (a set of articles from {\em Science}) better than LDA\cite{2007}. It is hence expected that a model based on the CTM will be able to infer pathway crosstalks and so will bring an improvement in predictive power for drug-pathway relationships.

The aim of this project is to implement and test such a model. In addition to reusing the Bayesian network from the CTM, the model will be augmented with a prior distribution of known gene-pathway relationships. A method called Variational EM will be used to train this model. Since the original CTM paper does not mention how the trained model can perform inference, an inference algorithm will have to be developed. The model will be evaluated first on a synthetic corpus of drug gene expression data (split into training and test sets) and then on publically available datasets (CMAP\cite{CMap} for gene expression data and KEGG\cite{KEGG} for pathway data).

\section*{Starting Point}

I know Python and have some experience in working with the SciPy stack. Python has lots of facilities for data processing and while it is an interpreted language, NumPy uses native LAPACK/BLAS (linear algebra) libraries as a backend. This means that the performance of the linear algebra routines is on par with C.

Since this project uses some concepts from the Artificial Intelligence II course (see Key Concepts), I will have to familiarise myself with the course before it begins.

There exists an R package for the Correlated Topic Model, as well as a C implementation of the Model written by the authors of the original paper (\url{https://www.cs.princeton.edu/~blei/ctm-c/}). I will however only be able to use those as a reference at most: adding priors to the model will change its implementation. In addition, the implementations also don't have a way to use the fitted models for inference.

\section*{Substance and Structure of the Project}

\subsection*{Key Concepts}

The key concepts in this project are drawn from the Artificial Intelligence II course, including topics such as Bayesian Networks and inference, methods for training and evaluating classifiers and probability distribution transformation. This project, however, will go beyond the scope of the course, since the Markov Chain Monte Carlo approach for inference in Bayesian networks is intractable for the Correlated Topic Model and Variational Expectation Maximization has to be used.

\subsection*{Major work items}

\subsubsection*{Required reading and research}
This involves familiarizing myself with the Bayesian Networks using the Artificial Intelligence II lecture notes, understanding the Bayesian Network training methods that are used in Latent Dirichlet Allocation (Gibbs Sampling and Markov Chain Monte Carlo) and the Correlated Topic Model (Variational EM). I will also need to get a better understanding of the biological context around the project, including topics such as genes, pathways, how the drug gene expression data is obtained using microarrays and how it has to be preprocessed.

\subsubsection*{Developing a model}

Define the Variational EM equations behind the training of the model and come up with an inference algorithm for the fitted model.

\subsubsection*{Implementation and testing}

Implement the software to handle the datasets: reading the CMAP/KEGG data, generating test datasets. The CMAP is a raw dataset taken from cell lines and so will also need to be preprocessed. At this stage, the actual training and classification procedures for the model will be implemented in Python. 

During development, the model will be tested using a synthetic corpus of drug gene expression data, whereas the final evaluation will consist of running the model on the publically available drug gene expression data (CMAP) and comparing the results with the LDA approach (see Evaluation and Success Criteria).

\subsubsection*{Extensions}
There are multiple other priors that can be incorporated into the model to improve predictive power, for example, the correlation between drugs: the molecular structure of a drug can be treated as a set of functional groups and so drugs with similar sets of functional groups are likely to have similar effects.

In addition, the Variational EM approach only gives a maximum likelihood estimate of the hidden variables. The model could be given a full Bayesian treatment by using Hierarchical Bayesian modelling.

\section*{Evaluation and Success Criteria}

The main criterion for success is a working implementation of the model as per its specification. This will be evaluated as follows:
\begin{itemize} 
\item Generate a set of random pathways, pathway-gene memberships (a probability distribution over genes) and drug-pathway interactions (a probability distribution over pathways)
\item Use that to generate a corpus of gene expression data as per how the Correlated Topic Model assumes it's generated. Split the gene expression data into a training and a test set.
\item Train the model using the training set and a random subset of pathway-gene memberships to serve as a prior.
\item Test the model: use it to predict the pathway distribution for each drug in the test set and compare these distribution with the reference distributions.
\end{itemize}

Note that the successful achievement of this objective does not imply the model has any predictive power for real drug data. It only verifies that the model works as expected and there are no errors in the implementation.

\section*{Resource Declaration}

The data for pathways will be taken from the KEGG database and the drug gene expression data will be taken from CMAP, which are publically available.

I will be using my own laptop for main development, a quad-core machine with Windows and ArchLinux on it. The source code will be kept in a Git repository which will be regularly pushed to GitHub, as well as to my MCS filespace. As the KEGG and CMAP data is publically available, and the code used to read and prepare it for processing will be backed up, I do not expect to require to make backups of the actual datasets.

\section*{Timetable: Workplan and Milestones to be achieved.}

Planned starting date is 27/10/2014.

\begin{enumerate}

\item {\bf Michaelmas weeks 2-4} 

Perform required reading and research.

\item {\bf Michaelmas weeks 5-7} 

Design the model and define the equations used for inference and classification.

\item {\bf Michaelmas week 8-vacation} 

Implement the model.

\item {\bf Lent weeks 0-2} 

Generate a synthetic dataset and verify the implementation. Write the progress report. Rehearse and deliver a presentation to the overseeing group.

\textbf{Milestone:} a working implementation of the model.

\item {\bf Lent weeks 3-5} 

Obtain and explore the KEGG and the CMAP datasets, perform preprocessing of the data.

\item {\bf Lent weeks 6-8}

Use the datasets to train the model and evaluate it against the performance of the LDA model.

\item {\bf Easter vacation} 

If there is time, work on the extensions to the project. Start writing the dissertation.

\item {\bf Easter term weeks 0-2}

Finish writing a draft dissertation. Submit to the Director of Studies and the Supervisor for review.

\item {\bf Easter term week 3}

Incorporate comments from the reviews. Submit the dissertation.

\end{enumerate}

\bibliographystyle{unsrt}
\bibliography{bibliography}

\end{document}