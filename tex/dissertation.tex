\documentclass[12pt,a4paper,twoside,openright]{report}
\usepackage[pdfborder={0 0 0}]{hyperref}    % turns references into hyperlinks
\usepackage[margin=25mm]{geometry}  % adjusts page layout
\usepackage{graphicx}  % allows inclusion of PDF, PNG and JPG images
\usepackage{verbatim}
\usepackage{docmute}   % only needed to allow inclusion of proposal.tex
\usepackage[noend]{algpseudocode}
\usepackage[]{algorithm}
\usepackage{amsmath}
\usepackage{mathtools}
\usepackage{enumitem}
\newcommand{\var}[1]{{\operatorname{\mathit{#1}}}}

\raggedbottom                           % try to avoid widows and orphans
\sloppy
\clubpenalty1000%
\widowpenalty1000%

\renewcommand{\baselinestretch}{1.1}    % adjust line spacing to make
                                        % more readable

\begin{document}

\bibliographystyle{plain}


%%%%%%%%%%%%%%%%%%%%%%%%%%%%%%%%%%%%%%%%%%%%%%%%%%%%%%%%%%%%%%%%%%%%%%%%
% Title


\pagestyle{empty}

\rightline{\LARGE \textbf{Artjoms Iskovs}}

\vspace*{60mm}
\begin{center}
\Huge
\textbf{Predicting drug-pathway interactions using the Correlated Topic Model} \\[5mm]
Computer Science Tripos -- Part II \\[5mm]
Trinity College \\[5mm]
\today  % today's date
\end{center}

%%%%%%%%%%%%%%%%%%%%%%%%%%%%%%%%%%%%%%%%%%%%%%%%%%%%%%%%%%%%%%%%%%%%%%%%%%%%%%
% Proforma, table of contents and list of figures

\pagestyle{plain}

\chapter*{Proforma}

{\large
\begin{tabular}{ll}
Name:               & \bf Artjoms Iskovs                       \\
College:            & \bf Trinity College                     \\
Project Title:     & \bf Predicting drug-pathway interactions using the Correlated Topic Model\\
Examination:        & \bf Computer Science Tripos -- Part II, July 2015  \\
Word Count:         & \bf ??\\
Project Originator: & N.~Pratanwanich                    \\
Supervisor:         & N.~Pratanwanich                    \\ 
\end{tabular}
}

\section*{Original Aims of the Project}
Implement the Correlated Topic Model for the problem of inferring pathways from drug gene expression data. Find a way to enforce gene-pathway membership priors so that the recovered topic structure can be related to the real data.

\section*{Work Completed}

\section*{Special Difficulties}
None
 
\newpage
\section*{Declaration}

I, Artjoms Iskovs of Trinity College, being a candidate for Part II of the Computer
Science Tripos, hereby declare that this dissertation and the work described in it are my own work,
unaided except as may be specified below, and that the dissertation does not contain material that has already been used to any substantial
extent for a comparable purpose.

\bigskip
\leftline{Signed [signature]}

\medskip
\leftline{Date [date]}

\tableofcontents

\listoffigures

\newpage
\section*{Acknowledgements}


%%%%%%%%%%%%%%%%%%%%%%%%%%%%%%%%%%%%%%%%%%%%%%%%%%%%%%%%%%%%%%%%%%%%%%%
% now for the chapters

\pagestyle{headings}

\chapter{Introduction}

Drugs are often investigated by using gene expression microarray analysis: a matrix of various samples of DNA is treated with different drugs and their expression is compared with the baseline expression.

Functionally similar genes are grouped into pathways.

Topic models come from the realm of natural language processing and are bag-of-words models that allow for exploratory data analysis on collections of documents. They assume that documents in a corpus are generated by first drawing a distribution of topics for every document, then picking words for a document by first choosing a topic from this distribution and then sampling a word from that distribution.

Using topic models in order to predict pathways affected by drugs has many benefits. Firstly, pathways that a new drug is likely to affect can be predicted without having to perform tests on real people, which means that useless drugs can be rejected quicker. In addition, this kind of analysis can allow researchers to find new uses (affected pathways) for existing drugs. Finally, the inferred correlation matrix can provide useful insights into relationships between pathways (such as disease comorbidity).

\chapter{Preparation}

\chapter{Implementation}

\section{Correlated Topic Model}

Bayesian plate notation here + explanation

\section{Training process}

Quote the derivation from Blei's paper?

The training process for the model is Bayesian inference: we express P(data|model) and vary the model parameters in order to maximize this probability

Probably a feedback picture with variational parameters and model parameters here.
\documentclass{article}
\usepackage[noend]{algpseudocode}
\usepackage[]{algorithm}
\usepackage{amsmath}
\usepackage[cm]{fullpage}
\newcommand{\var}[1]{{\operatorname{\mathit{#1}}}}

\begin{document}
\begin{algorithm}
   \caption{Variational Inference for a document}
   \label{alg:variational_inference}
   \Comment{$\var{mod-params}$ is a tuple $(\mu, \Sigma, \beta), \var{var-params}$ is a tuple $(\zeta, \phi, \lambda, \nu^2)$.}
   \begin{algorithmic}[1]
      \Function{optimize-zeta}{$\lambda, \nu^2$}
      \State \Return $\sum{}{e^{\lambda + \frac{1}{2}\nu^2}}$
      \EndFunction 
      \Function{optimize-lambda}{$\var{var-params}, \var{mod-params}, w, n$}
      \State The objective functions use the $\lambda$ given in the function parameters instead of the one in $\var{var-params}$.
    	  \Function{objective-lambda}{$\lambda, \var{var-params}, \var{mod-params}, w, n$}
		      \State\Return $\frac{1}{2}(\lambda - \mu)^T\Sigma^{-1}(\lambda-\mu) - \sum{}{\phi} + \frac{N}{\zeta}\sum{}{e^{\lambda + \frac{1}{2}\nu^2}}$
	      \EndFunction
     	  \Function{objective-lambda-gradient}{$\lambda, \var{var-params}, \var{mod-params}, w, n$}
		      \State\Return $\Sigma^{-1}(\lambda - \mu) - \sum{}{\phi} + \frac{N}{\zeta}\sum{}{e^{\lambda + \frac{1}{2}\nu^2}}$
	      \EndFunction
	  \State $\lambda\gets $ result from a SciPy minimizer using functions \Call{objective-lambda}{} and \Call{objective-lambda-gradient}{}
      \State \Return $\lambda$
      \EndFunction
      \Function{optimize-nu-sq}{$\lambda, \nu^2$}
      
	\State The objective functions use the $\nu^2$ given in the function parameters instead of the one in $\var{var-params}$
    	  \Function{objective-nu-sq}{$\nu^2, \var{var-params}, \var{mod-params}$}
		      \State\Return $trace(\frac{1}{2}diag(\nu^2)\Sigma^{-1}) + \frac{N}{\zeta}\sum{}{e^{\lambda + \frac{1}{2}\nu^2}} - \frac{1}{2}\sum{}{(1 + ln(2\pi \nu^2))}$
	      \EndFunction
     	  \Function{objective-nu-sq-gradient}{$\nu^2, \var{var-params}, \var{mod-params}$}
		      \State\Return $\frac{1}{2}diag(\Sigma^{-1}) + \frac{N}{2\zeta}e^{\lambda + \frac{1}{2}\nu^2} - \frac{1}{2\nu^2}$
	      \EndFunction
	  \State $\nu^2\gets $ result from a SciPy minimizer using functions \Call{objective-nu-sq}{} and \Call{objective-nu-sq-gradient}{}
      \State \Return $\nu^2$
      \EndFunction
      \Function{optimize-phi}{$\var{var-params}, \var{mod-params}, w, n$}
		\State $\phi_{n,i}\gets e^{\lambda_i}\beta_{i, w_n}n_n$
	  \State Normalize $\phi$ so that every row sums up to 1
      \State \Return $\phi$
      \EndFunction
   
      \Function{variational-inference}{$\var{mod-params}, w, n$}
      \State Initialize the variational parameters: $\var{var-params}=(\zeta=10, \phi=1, \lambda=0, \nu^2=1$)
      \While{change in likelihood bound $>$ threshold}
      \State $\zeta\gets \Call{optimize-zeta}{\lambda, \nu^2}$
      \State $\lambda\gets \Call{optimize-lambda}{\var{var-params}, \var{mod-params}, w, n}$
      \State $\zeta\gets \Call{optimize-zeta}{\lambda, \nu^2}$
      \State $\nu^2\gets \Call{optimize-nu-sq}{\var{var-params}, \var{mod-params}}$
      \State $\zeta\gets \Call{optimize-zeta}{\lambda, \nu^2}$      
      \State $\phi\gets \Call{optimize-phi}{\var{var-params}, \var{mod-params}, w, n}$
      \EndWhile
    \State \Return $\var{var-params}$
    \EndFunction
\end{algorithmic}
\end{algorithm}

\begin{algorithm}
	\caption{Model parameter inference for the whole corpus}
	\label{alg:expectation_maximisation}
	\begin{algorithmic}[1]
		\Function{inference}{$\var{corpus}, K$}
			\State Initialize the model parameters: $\var{mod-params}=(\mu = 0, \Sigma=I, \beta=\var{priors})$
			\While{change in likelihood bound $>$ threshold}
				\State $\var{var-params}\gets \Call{variational-inference}{\var{mod-params}, w, n}$ for all $(w, n)$ in $\var{corpus}$
				
				\State $\mu \gets \frac{1}{D}\sum\limits_{v \in \var{var-params}}{v.\lambda}$
				\State $\Sigma \gets \frac{1}{D}\sum\limits_{v \in \var{var-params}}{(diag(v.\nu^2) + (v.\lambda - \mu)(v.\lambda - \mu)^T)}$
				\State $\beta \gets 0$
				\For{$d$ in $1..D$}
					\For{$w$ in $1..len(corpus_d.w)$}
						\For {$i$ in $1..K$}
							$\beta_{i, w} \gets \beta_{i, w} + corpus_d.n_w \times \var{mod-params}_d.\phi_{w, i}$
						\EndFor
					\EndFor
				\EndFor
				\State Normalize $\beta$ so that every row sums up to 1
			\EndWhile
			\State\Return $\var{mod-params}$
		\EndFunction
	\end{algorithmic}
\end{algorithm}

\end{document}

The core of the training process is optimizing the likelihood bound on the model: how likely is the corpus given the model. By altering the parameters of the model in order to maximize this bound, the model is trained on the data.

The training happens in cycles. First, a variational distribution is fit to every document in the corpus. Then, the new model parameters are inferred and the likelihood bound recalculated. This process repeats until changes in the likelihood bound are negligible.

\section{Incorporating priors into the Model}

In the training process, observe that:

This means that a zero entry in the $\beta$ matrix that the training process is initialized with will carry through to the variational parameters for every document and will be fed back into the updated $\beta$, thus enforcing that the inferred per-topic word distributions have zeros in the set places. This has the effect of setting pathway-gene membership priors and ensures that the inferred topic structure is referring to the actual pathways, so the results of the inference process will allow us to make judgements about real-world phenomena.

\section{Classification process}

TODO: read up on how theta is inferred

\section{Implementation in code}

Python!

\section{Code optimization}

Since Python is an interpreted language, it is great for quick prototyping and implementation. Unfortunately, it means that its performance is several orders of magnitude below that of C code. Using NumPy for numerical calculations definitely helps, since the matrix product operations are performed in native code, but I still had performance issues which would have made working with the actual KEGG/CMap datasets very inconvenient. The training time for 1/10th of the dataset was about 3-4 hours, which implies more than 30-40 hours for the full dataset, since increasing its size would worsen the convergence rate. Using the profiler, I identified several performance bottlenecks in the code.

Firstly, the EM process used explicit Python list comprehensions that are evaluated using the Python interpreter, which I converted into matrix products.

Another major bottleneck is the likelihood bound calculation, which has to be performed every variational inference cycle for every document, since the termination criterion is based on the magnitude of changes in the bound. One solution to this would be not calculating the likelihood bound every iteration, but this resulted in worse performance: another variational inference cycle was more expensive than checking the bound. I used \texttt{numexpr}, a package that compiles operations on \texttt{NumPy} arrays into its own internal virtual machine bytecode and supports multithreaded computations on arrays.

The variational inference process requires calling the \texttt{SciPy} function minimizer for some parameters multiple times, which requires an initial value to start its search from. It turned out that in the beginning of the variational inference (every EM iteration) the initial values were the default ones. Recycling the variational parameters to continue the search from the previous iteration gave yet another performance improvement.

I also recompiled my \texttt{NumPy} distribution to use the Intel Math Kernel Library as its backend. Free for academic and personal use, Intel MKL provides a linear algebra library that implements the BLAS (Basic Linear Algebra Subsystem) API, optimized for manycore Intel CPUs.

Since the variational inference process is independent for every document, it can be parallelized across multiple cores. However, because of the Python Global Interpreter Lock, only one thread can execute Python bytecode at a time. This meant that I had to launch several Python processes if I wanted to take advantage of my CPU's capabilities. While this did provide an increase in CPU usage, it resulted in worse performance than using \texttt{numxepr} and Intel MKL's natural multithreading capabilities.

These efforts brought the training time down to about 8-9 hours for the whole dataset.

\chapter{Evaluation}

\section{Toy datasets}

While the model could be used to predict real-world drug-pathway interactions, it is not at all certain that they actually do follow the generative framework defined by the CTM. Hence, most of the evaluation of the model was done on generated datasets which allowed to verify the model actually corresponds to the specification as well as explore the capabilities of the CTM.

The generative process for a toy dataset proceeds as follows:

\begin{itemize}[noitemsep]
\item Sample $\mu$ from a K-dimensional uniform distribution
\item Sample $\beta$ from K voclen-dimensional uniform distributions and normalize it
\item Sample $\Sigma$ from an inverse Wishart distribution
\item Generate N random documents
\end{itemize}

To generate a single document:

\begin{itemize}[noitemsep]
\item Sample $\eta_d$ from $N(\mu, \Sigma)$ and normalize it to get the topic distribution for the document, $\theta_d = \frac{e^{\eta_d}}{\sum{e^{\eta_d}}}$
\item Repeat W times where W is the number of words in each document:
\begin{itemize}[noitemsep]
\item Sample $z_{d, n}$, the topic the word belongs to, from $\eta_d$
\item Sample $w_{d, n}$ from $\beta{z_{d, n}}$, the word distribution for that topic
\end{itemize}
\end{itemize}

W is chosen to be suffciently large as to provide a good approximation to the distribution of words in each document, since this is what essentially is being sampled.

The generation process returns the model parameters, the counts of words in each document, as well as the topic distribution for every document. Only the word counts are fed into the model: the other outputs are used for evaluation.

The following subsections describe the datasets which were generated and goes into detail about the methods of evaluation that were performed.

\subsection{Description of toy datasets}

The datasets were generated so that they mimic the properties of the real world KEGG/CMap dataset. All datasets have the following in common:

\begin{itemize}[noitemsep]
\item Vocabulary length (number of genes): 3000
\item Number of documents (drugs): 500
\item Number of words in each document: 1000
\item Density of $\beta$: 0.0138
\item Density of $\theta$: 0.1231
\end{itemize}

Densities here refer to the fraction of non-zero elements in each matrix. The KEGG/CMap dataset is very sparse: each pathway only affects, on average 1.38\% of all genes and each drug only affects, on average, 12.3\% of all pathways. The density value for $\beta$ was calculated by taking the matrix of priors (since it explicitly enforces the pattern of zeros in the result), whereas the density value for $\theta$ was calculated by taking the reference (CTD) dataset.

With these parameters, 5 datasets were generated:

\begin{itemize}[noitemsep]
\item 10 topics
\item 20 topics
\item 50 topics
\item 100 topics
\end{itemize}

These datasets allow to investigate how the performance of the model changes when the number of topics increases.

The parameters of the fifth dataset were designed to mimic the real-world one as closely as possible in order to determine, by comparing the performance of the model on that dataset against the real-world one, whether the drug gene expression data follows the generative framework described by the CTM. In addition to reusing all the parameters the first 4 datasets were generated with, for this dataset I also approximated the $\mu$ and $\sigma$ of the reference by first calculating the topic distribution $\theta$ implied by the dataset (assuming all pathways in the resultant matrix are equally weighted) and then calculating the mean and the covariance matrix of the distribution of pathways based on it.

In the last 4 datasets, I varied the density of the covariance matrix: the proportion of pathways in the implied matrix of correlations that are connected:

\begin{itemize}[noitemsep]
\item No correlations: use the identity matrix as $\Sigma$
\item Density 10\%
\item Density 50\%
\item Density 90\%
\end{itemize}

\subsection{Convergence analysis}

At every iteration, the model outputs a likelihood bound that it assigns to the data and continues updating the model parameters until the likelihood bound converges (the change is less than a certain threshold). The likelihood bound values over iterations are plotted here.

It can be observed that increasing the number of topics decreases the likelihood bound the model assigns to the data andmakes convergence slower.

\subsection{Evaluating the topic prediction}

Since the final output of the model is a distribution of topics for every document, it makes sense to evaluate the model based on that. However, the big problem in the evaluation of the model is topic identifiability: the recovered topics are not at all guaranteed to be in the correct order and if they are close together, they are not guaranteed to be recovered. This means that even if the model gave perfect predictions about every document, it could score poorly simply because a topic that has one index in the model is referred to by a different index in the evaluation dataset.

The performance measure that is not affected by the topic identifiability is the document similarity matrix. Given the inferred and the reference $\theta$, the per-document topic distribution, we can take their cosine similarity to see how similar any two documents are. Thus, for the whole corpus, we can construct a document similarity matrix and the two matrices for the inferred and the reference distributions can be compared:

\begin{equation}
M_{i,j} = \frac{\theta_i \cdot \theta_j}{|\theta_i||\theta_j|}
\end{equation}

To gain access to more evaluation measures, a similar matrix can be constructed for $\beta$, the per-topic word distribution. Using this matrix, we can see which topics from the inferred structure are the closest to the original ones and permute the results of the model accordingly. However, this doesn't help with some generated topics being close together and so the generated dataset actually has some entries in the word-topic matrix set to zero. In addition, the model is trained with the pattern of zeros set as its prior, which improves the chances of a successful topic recovery. While this can be considered cheating, the real-world gene-pathway membership dataset is, in fact, very sparse as well (for every pathway, most genes don't participate in it), so this is a necessary measure.

\subsection{Evaluating the parameter recovery}

The benefit of evaluating the model with toy datasets is that the model parameters the reference dataset was generated with are available. Hence, they can be compared to the parameters inferred by the model. Topic identifiability is still an issue here, since each one of the model parameters references topics in some way: $\mu$ is the mean proportion of topics, $\Sigma$ is the covariance matrix of the topic proportions and $\beta$ is the topic-word distribution. Hence, just like in the evaluation of the topic inference, the entries in the parameters have to be permuted as per the ``map" that relates the model's and the generated data's view of topics:

\begin{align*}
&\mu'_{map_i} = \mu_i \\
&\Sigma'_{map_i, map_j} = \Sigma_{i, j} \\
&\beta'_{map_i, w} = \beta_{i, w}
\end{align*}

TODO: add about mu normalisation?

In addition, the covariance matrix $\Sigma$ is turned into the correlation matrix (normalized) so that the correlations inferred by the model and the reference correlations can be compared:

\begin{equation}
C_{i, j} = \frac{|\Sigma^{-1}_{i, j}|}{\sqrt{\Sigma^{-1}_{i, i}\Sigma^{-1}_{j, j}}}
\end{equation}

The parameters are then compared to the reference using the normalized root-mean-square error (RMSE):

\begin{equation}
\mathit{RMSE}(I, R) = \frac{\sqrt{\sum{(I-R)^2}}}{\sqrt{D}(max(R) - min(R))}
\end{equation}

where $I$ is the inferred parameter, $R$ is the reference parameter and $D$ is the size of the matrix/vector.

TODO: add about graph recovery (correlation matrix)

\subsection{Evaluating the rank recovery}

The validation dataset (CTD) does not contain actual drug-pathway distributions: instead, for every drug, it lists the pathways that it does affect. Therefore, the toy dataset generation method had to be altered: when the topic distribution for a drug is drawn, a certain number of lowest-represented pathways is set to zero, enforcing that every drug, in essence, affects only a few pathways.

The problem of evaluating pathway distributions against a list of relevant pathways strikingly resembles a similar problem from information retrieval: given a ranking for documents (here, pathways) output by a model for various queries (here, drug gene expression profiles) and a judgement as to which documents (pathways) are relevant (actually affected by the drug), how can evaluation be performed?

Two metrics that are used in information retrieval to evaluate such systems are precision and recall: precision is the fraction of retrieved documents that are relevant and recall is the fraction of relevant documents that are retrieved. These metrics can be calculated at all the cutoff ranks in the output pathway list and used to plot a precision-recall curve.

Another way of evaluating the distributions I came up with is, for every drug and for every pathway that this drug actually affects, summing the probabilities predicted by the model for that pathway. This results in a value for every drug that represents, in essence, the fraction of drug-pathway activity that the model got ``right":

\begin{equation}
c_d = \sum\limits_{p \in \mathit{ref}_d}{\theta_{d, p}}
\end{equation}

where $\mathit{ref}_d$ is a set of correct pathways for every drug and $\theta_d$ is the predicted pathway distribution for drug $d$.

The distribution of these values can then be compared to the performance of random guessing to judge whether the model gives significantly different results.

\section{KEGG and CMap datasets}

What are KEGG/CMap?

Some diagrams and topic graphs here, compare with the LDA results, compare with the known drug pathways.

Cluster pathways/drugs

Side plot from LDA, heatmap of correctly predicted pathways

\chapter{Conclusion}

%%%%%%%%%%%%%%%%%%%%%%%%%%%%%%%%%%%%%%%%%%%%%%%%%%%%%%%%%%%%%%%%%%%%%
% the bibliography
\addcontentsline{toc}{chapter}{Bibliography}
\bibliography{bibliography}

%%%%%%%%%%%%%%%%%%%%%%%%%%%%%%%%%%%%%%%%%%%%%%%%%%%%%%%%%%%%%%%%%%%%%
% the appendices
\appendix

\chapter{Project Proposal}

\documentclass[12pt,a4]{article}
\usepackage{hyperref}
\usepackage{color}
\usepackage{fullpage}
\begin{document}

\vfil


\begin{flushright}
\large{A. I\v{s}kovs}\\
\texttt{ai280}\\
Trinity College
\end{flushright}

\vspace*{\fill}
\begin{center}{\large Computer Science Tripos, Part II Project Proposal}

\vspace{0.3in}
\textbf{\center{\Large Predicting drug-pathway interactions using the Correlated Topic Model}}

\vspace{0.4in}
\centerline{\large \today}
\end{center}
\vspace*{\fill}

\vfil

%
%\noindent{\bf Project Originator:} N. Pratanwanich
%\vspace{0.4in}
%
%\noindent
%{\bf Project Supervisor:} N. Pratanwanich\\
%
%\noindent
%{\bf Signature:}
%\vspace{0.2in}
%
%\noindent
%{\bf Director of Studies:} Dr A. C. Norman\\
%
%\noindent
%{\bf Signature:}
%\vspace{0.2in}
% 
%\noindent
%{\bf Project Overseers:} Dr~A.~V.~S.~Madhavapeddy  \& Dr~S.~H.~Teuffel\\
%
%\noindent
%{\bf Signatures:}

\pagebreak

% Main document

\section*{Introduction}

Latent Dirichlet Allocation\cite{Blei} is a bag-of-words topic modelling algorithm that treats documents in a corpus as being generated by picking a set of topics and then drawing words from those topics. Using a sampling method, the posterior distributions of unknown variables (probability distributions of topics and words within topics) are inferred. This allows, for an arbitrary document, to infer a probability distribution of topics which it is about.

The 2014 paper\cite{Pratanwanich2014} by Naruemon Pratanwanich and Pietro Lio describes a model based on this algorithm to predict drug-pathway relationships: differential gene expression profiles of drug treatments are treated as documents, pathways where genes are functionally grouped as topics and genes as words. The model had to be augmented with priors: the pathway-gene relationships for some pathways are already known. In the end, this method provided better predictive results of pathways activated by certain drugs than the state-of-the art methods.

A problem with Latent Dirichlet Allocation and hence this approach is that it assumes that the topics (or pathways) are uncorrelated. This assumption is usually not true: a document about Computer Science is more likely to be about Mathematics than, say, Geology. Another 2014 paper\cite{C4MB00014E} by Naruemon Pratanwanich and Pietro Lio presents a model that assumes pathway crosstalk (correlation). It uses matrix factorisation together with this assumption to predict pathway responsiveness for drugs that affect multiple pathways and finds that using the correlation assumption results in a better fit to the gene expression data.

The Correlated Topic Model\cite{2007} was proposed as an improvement on the Latent Dirichlet Allocation: instead of using the Dirichlet distribution to model the relative proportions of topics in a document, it uses the logistic normal distribution. The parameters of the distribution are the mean and the covariance matrix, which allows for modelling correlations between topics. The original paper\cite{2007} uses a deterministic approach called Variational EM (Expectation Maximization) to train this model (due to the non-conjugacy of the logistic normal, the Markov Chain Monte Carlo approach with Gibbs Sampling is intractable). The Correlated Topic Model fit the sample corpus (a set of articles from \textit{Science}) better than LDA\cite{2007}.

Since the matrix factorization model that assumed pathway correlation\cite{C4MB00014E} and the LDA method\cite{Pratanwanich2014} outperform matrix factorization without correlations, it is suspected that adapting the CTM to the problem of prediction of pathway responsiveness to drug treatment would yield another improvement in predictive power.

The aim of this project is to implement and test such a model. In addition to reusing the Bayesian network from the CTM, the model will be augmented with a prior distribution of known gene-pathway relationships. The equations for training the model will be derived, based on the existing CTM equations for Variational EM\cite{2007}. The model will be tested on a synthetic corpus of drug gene expression data for model verification and then on publically available datasets (CMAP\cite{CMap} for gene expression data and KEGG\cite{KEGG} for pathway data). The results will be compared with the latent variables inferred by the LDA approach. While it's not possible to guarantee that the model will perform better in every case, the results will be investigated to see in which cases the LDA approach performs better than CTM and vice versa.

\section*{Starting Point}

I know Python and have some experience in working with the SciPy stack. Python has lots of facilities for data processing and while it is an interpreted language, NumPy uses native LAPACK/BLAS (linear algebra) libraries as a backend. This means that the performance of the linear algebra routines is on par with C.

Since this project uses some concepts from the Artificial Intelligence II course (see Key Concepts), I will have to familiarise myself with the course before it begins.

There exists an R package for the Correlated Topic Model, as well as a C implementation of the Model written by the authors of the original paper (\url{https://www.cs.princeton.edu/~blei/ctm-c/}). I will however only be able to use those as a reference at most: adding priors to the model will change its implementation. On the other hand, the C implementation has a way to perform posterior inference (prediction) on new documents, which can help with one of the extensions to the project.

\section*{Substance and Structure of the Project}

\subsection*{Key Concepts}

The key concepts in this project are drawn from the Artificial Intelligence II course, including topics such as Bayesian Networks and inference, methods for training and evaluating classifiers and probability distribution transformation. This project, however, will go beyond the scope of the course, touching on Variational Inference, conjugate and non-conjugate distributions and Expectation Maximization.

\subsection*{Major Work Items}

\subsubsection*{Required Reading and Research}

\begin{itemize}
\item Artificial Intelligence II Lecture notes: introduction to Bayesian networks and inference
\item Chapters 9 and 10 of Christopher Bishop's book\cite{Bishop:2006:PRM:1162264} on computer vision models: Mixture models (of which LDA and CTM are variations) and the Expectation Maximization algorithm, as well as Variational Inference.
\item ``Build, Compute, Critique, Repeat: Data Analysis with Latent Variable Models"\cite{doi:10.1146/annurev-statistics-022513-115657}, an introduction to latent variable models that describes how to train the models using mean-field Variational Inference, use them to perform predictions and evaluate them.
\item The original LDA\cite{Blei} and CTM\cite{2007} papers to understand the structures of the respective generative probabilistic models and their inference based on Variational EM.
\item Naruemon Pratanwanich and Pietro Lio's paper\cite{Pratanwanich2014}  on adapting LDA in order to familiarize myself with how microarray data is preprocessed into a pseudo-drug ``document", how the gene-pathway membership priors are incorporated into the model, as well as the ranking method used to evaluate its performance.

I will also need to get a better understanding of the biological context around the project, including topics such as genes, pathways, how the drug gene expression data is obtained using microarrays and how it has to be preprocessed.

\end{itemize}

\subsubsection*{Developing a Model}

The classic Expectation Maximization algorithm consists of two steps:

\begin{itemize}
\item \textbf{E}: Calculate the expected value of the log likelihood of seeing the latent variables given the observed variables and the model parameters.
\item \textbf{M}: Maximize this function by updating the model parameters.
\end{itemize}

The CTM paper\cite{2007} applies an algorithm called Variational EM that, using a variational distribution, approximates the posterior distributions of latent variables with respect to the model parameters in the \textbf{E} step and estimates the model parameters with respect to variational parameters in the \textbf{M} step.

I have to understand and alter this algorithm in order add prior information on gene-pathway memberships to the model.

\subsubsection*{Implementation and Testing}

In Python, implement:

\begin{itemize}
\item The software to generate synthetic data for model verification and to preprocess real datasets.
\item The actual Correlated Topic Model that incorporates the priors.
\end{itemize}

During development, the model will be tested using toy datasets of synthetic gene expression and pathway data. Using the real datasets, it will then be compared against the LDA method (see Evaluation and Success Criteria).

\subsubsection*{Fallback Plan}
In case I fail to incorporate priors into the Correlated Topic Model, I will use one of the available libraries for fitting the CTM and instead work on processing the gene data into a drug "document" that can be understood by the library, as well as developing a way to predict pathway distributions for a previously unseen drug. The evaluation procedure in that case will have to be changed, since not having priors in the model will mean that the pathways will be completely arbitrary and will not correspond to real drug data.

Another option is using a less complicated algorithm such as Random Forests that has high predictive power on drug-pathway perturbations\cite{Riddick15012011}. In that case the setting will be that of supervised learning, where the training data are known pairs of drugs and pathways and the features are the differential gene expression data. However, in this case I won't be able to test the hypothesis of pathway correlation and how the data are generated as I would be able to by using the aforementioned generative probabilistic models.

\section*{Evaluation and Success Criteria}

The \textbf{main criterion} for success is a working implementation of the model as per its specification. This will be evaluated as follows:
\begin{itemize} 
\item Generate a toy dataset of random pathways, pathway-gene memberships (a probability distribution over genes) and drug-pathway interactions (a probability distribution over pathways)
\item Use that to generate a corpus of gene expression data as per how the Correlated Topic Model assumes it's generated.
\item Together with a random subset of the pathway-gene membership data to serve as a prior, use the model to infer the latent variables and compare them with the ones that generated the dataset.
\end{itemize}

Note that the successful achievement of this objective does not imply the model has any predictive power for real drug data. It only verifies that the model works as expected and there are no errors in the implementation.

The comparison with the LDA method can be performed in two ways. Firstly, the model can be run on the CMAP/KEGG datasets to infer the latent variables. One of these is the per-drug pathway distribution, which can then be compared with the distribution inferred by the LDA method based on the accuracy on the reference data. It is possible that the developed model will perform better on some drugs than LDA and worse on others. This will be investigated in order to know which types of drugs it's better to use one model over another.

In case \textbf{the extension goal} of having a method for predicting pathways activated by certain drugs has been reached, the model will be tested on reference data (known drug-pathway relationships). This will be done by performing the predictions for known drugs and then using the pathway ranking metric described in\cite{Pratanwanich2014}.


\section*{Extensions}
The original CTM paper does not explicitly mention how the CTM can be used to calculate the topic proportions for a new document. One extension would hence involve performing that, in which case it will be possible to better compare the performance of the model with the LDA approach using the ranking method described in the paper\cite{Pratanwanich2014} (without this extension, we can still compare the results by considering the latent variables that both models inferred from the data).

Yet another extension can be making an interface for the trained model in order to create a stand-alone tool that researchers can use.

There are multiple other priors that can be incorporated into the model to improve predictive power, for example, the correlation between drugs: the molecular structure of a drug can be treated as a set of functional groups and so drugs with similar sets of functional groups are likely to have similar effects.

In addition, the Variational EM approach only gives a maximum likelihood estimate of the model parameters. The model could be given a full Bayesian treatment by using hierarchical Bayesian modelling on the model parameter. This may help solve the known problem of singularity of the covariance matrix when there is a large number of topics\cite{Masada:2013:RIC:2525761.2525819}.

\section*{Resource Declaration}

The data for pathways will be taken from the KEGG\cite{KEGG} database and the drug gene expression data will be taken from CMAP\cite{CMap}, which are publically available.

I will be using my own laptop for main development, a quad-core machine with Windows and ArchLinux on it. The source code will be kept in a Git repository which will be regularly pushed to GitHub, as well as to my MCS filespace. As the KEGG and CMAP data is publically available, and the code used to read and prepare it for processing will be backed up, I do not expect to require to make backups of the actual datasets.

\section*{Timetable: Workplan and Milestones to be achieved.}

Planned starting date is 27/10/2014.

\begin{enumerate}

\item {\bf 27/10/2014 -- 09/11/2014} 

Perform required reading and research (as per the Major Work Items section). Set up the outline for the dissertation.

\item {\bf 10/11/2014 -- 23/11/2014} 

Modify the model and adapt the equations used for inference to incorporate priors. Start writing the Method section of the dissertation.

\item {\bf 24/11/2014 -- 11/01/2015} 

Implement the model in Python. Generate a small toy dataset to verify the implementation. Debug the code.

\item {\bf 12/01/2015 -- 25/01/2015}

Work on optimizing and debugging the code so that it can run in reasonable time on a large toy dataset. Write the progress report. Continue writing the Method and start writing the Evaluation sections of the dissertation.

\textbf{Milestone:} a working implementation of the modified model. 

\item {\bf 26/01/2015 -- 08/02/2015} 

Submit the progress report. Rehearse and deliver a presentation to the overseeing group.

Obtain and explore the KEGG and the CMAP datasets. Preprocess the data into a suitable format for the model.

\item {\bf 09/02/2015 -- 22/02/2015}

Derive the equations required for prediction of pathways that are perturbed given a new drug (extension 1).

\item {\bf 23/02/2015 -- 08/03/2015}

Use the CMAP/KEGG datasets to train the model and evaluate it against the performance of the LDA model.

\item {\bf 09/03/2015 -- 19/04/2015}

Work on the remaining extensions to the project (such as packaging the model as a stand-alone tool) and continue writing the dissertation.

\item {\bf 20/04/2015 -- 03/05/2015}

Finish writing a draft dissertation. Submit to the Director of Studies and the Supervisor for review, allowing 2 weeks to read the draft.

\item {\bf 04/05/2015 -- 10/05/2015}

Incorporate comments from the reviews. Submit the dissertation.

\end{enumerate}

\bibliographystyle{unsrt}
\bibliography{bibliography}

\end{document}

\end{document}
